\documentclass[12pt,a4paper,final]{article}
\usepackage[utf8x]{inputenc}
\usepackage{ucs}
\usepackage[francais]{babel}
\usepackage[T1]{fontenc}
\usepackage{makeidx}
\usepackage{graphicx}
\usepackage{tikz}

\author{François \textsc{Hallereau} \\ Benjamin \textsc{Sientzoff}}
\title{GlossyGlossaire}

\makeindex

\begin{document}

\maketitle

\vspace{5cm}

\tableofcontents

\newpage

\section*{Introduction}
\addcontentsline{toc}{section}{\protect\numberline{}Introduction}
Ce projet consiste à implémenter un dictionnaire (structure de données non ordonnée avec unicité). Cette implémentation a été réalisé de deux manières différentes. 

L'une utilise une table de hachage dans laquelle on stocke les mots. L'autre est un arbre où sont stocké un à un les caractères des mots à sauvegarder.

\section{Le \emph{dictionnache}}
Cette version reprend la table de hachage du précédent tp.

Pour cette implémentation, les fonctions d'ajout et de suppression reprennent les fonctions déjà existantes.
 


\section{L'\emph{Arbramots}}

\section*{Conclusion}
\addcontentsline{toc}{section}{\protect\numberline{}Conclusion}


\end{document}
